\documentclass{scrartcl}
\usepackage[ngerman]{babel}
\usepackage[utf8]{inputenc}
\usepackage{xcolor}
\usepackage{longtable}

\usepackage[colorlinks=true,urlcolor=blue,linkcolor=purple]{hyperref}

\newcommand\email[1]{\texttt{#1}}
\title{Dokumentation zum Programm iopattern2dios}
\author{Daniela Koudela\\ ZIH\\ Technische Universität Dresden}
\date{\today}

\begin{document}
\maketitle
\tableofcontents
% folgende Kapitel sind noch zu bearbeiten:
% - Benutzung von iopattern/rabbit
\section{Wofür ist das Programm gut?} % erste Version fertig
Das Programm \emph{iopattern2dios} nimmt die Ausgabe von 
\emph{iopattern/rabbit} 
und wandelt sie in ein XML-Format um, welches als Eingabe für das Programm 
\emph{DIOS} verwendet werden kann.

\subsection{Was macht das Open Trace Format (OTF)} % erste Version fertig
Mit dem \emph{Open Trace Format} (OTF) werden Trace-Informationen während dem 
Laufen von massiv parallelen Programmen gesammelt, um das Programmverhalten
mittels geeigneter Analyseprogramme detailliert untersuchen zu 
können.\footnote{Für weitere Informationen zu OTF siehe
\url{http://www.tu-dresden.de/die_tu_dresden/zentrale_einrichtungen/zih/forschung/software_werkzeuge_zur_unterstuetzung_von_programmierung_und_optimierung/otf}}

% Was macht iopattern/rabbit?
\subsection{Was macht iopattern/rabbit?} % erste Version fertig
Das Programm \emph{iopattern/rabbit} analysiert und klassifiziert I/O-Aufrufe
innerhalb einer OTF-Datei.

Um das Programm \emph{iopattern/rabbit} zu erhalten, schickt man am besten eine
entsprechende Anfrage an Michael Kluge (\email{michael.kluge@tu-dresden.de}).

% Was macht DIOS?
\subsection{Was macht DIOS?} % erste Version fertig
DIOS ist ein in C++ geschriebener Dateisystem-Benchmark, 
um I/O-Aktivitäten einer Anwendung zu wiederholen
Die Input-Datei muss im XML-Format sein. Diese beschreibt, welche 
I/O-Aktivitäten jeder Prozess ausführt.\footnote{Weitere Details sind in 
der Dokumentation von \emph{DIOS} erklärt. Diese befindet sich in der 
\emph{DIOS}-Installation im Verzeichnis \texttt{docu/latex}.}

Um das Programm \emph{DIOS} zu erhalten, schickt man am besten eine 
entsprechende Anfrage an Michael Kluge (\email{michael.kluge@tu-dresden.de}).
%%% Installation %%%
\section{Installation} % erste Version fertig
Damit man eine Trace-Datei in XML umwandeln kann, muss man OTF und 
\emph{iopattern/rabbit} installieren. Hat man bereits die Ausgabe von 
\emph{iopattern/rabbit}, so ist keine Installation der beiden Programme nötig. 

\emph{iopattern2dios} ist ein python-Programm und muss somit nicht groß 
installiert werden. Es reicht, die Datei \texttt{iopattern2dios.py} 
in ein geeignetes Verzeichnis zu legen.
\subsection{Installation von OTF} % erste Version fertig
Zuerst muss OTF-1.11 aus dem Internet 
(\url{http://www.tu-dresden.de/die_tu_dresden/zentrale_einrichtungen/zih/forschung/software_werkzeuge_zur_unterstuetzung_von_programmierung_und_optimierung/otf}) 
heruntergeladen werden. Man entpackt es in einem geeigneten Verzeichnis. 
Zusätzlich braucht man noch das devel-Paket der libz. Dazu muss das Paket
\texttt{zlib1g-dev} installiert werden. Dann 
wechselt man in das neu entstandene Verzeichnis \texttt{OTF-1.11.2goldfish}
und installiert das Programm entsprechend den Anweisungen aus der Datei
\texttt{INSTALL}. Dabei muss beachtet werden, dass die Anweisung 
\texttt{make install} durch \texttt{sudo make install} ersetzt werden muss,
damit es funktioniert.

\subsection{Installation von iopattern/rabbit} % erste Version fertig
Man folgt der Anleitung, die sich in der Datei \texttt{INSTALL} im Verzeichnis
\texttt{iopattern} befindet. Eventuell muss man anstatt \texttt{./configure}
\begin{verbatim}
   ./configure --with-otf-root="/usr/local"
\end{verbatim}
eingeben, um \emph{iopattern/rabbit} zu sagen, wo sich OTF befindet.
%%% Benutzung %%%
\section{Benutzung} % zu bearbeiten
%% Benutzung von iopattern/rabbit %%
\subsection{Benutzung von iopattern/rabbit} % zu bearbeiten
Als Eingabe braucht \emph{iopattern/rabbit} ein von \emph{VampirTrace} 
erzeugtes Trace. Dieses besteht aus einer globalen Datei mit der 
Endung \texttt{.otf},
einer Definitionsdatei mit der Endung \texttt{.def.z} und für jeden Prozeß
eine Datei, die die mit \emph{libz} komprimierte Programmspur im OTF-Format
enthält.

Um das Programm \emph{iopattern/rabbit} zu starten, ruft man das Programm 
\emph{rabbit} im Verzeichnis \texttt{iopattern/rabbit} auf und übergibt 
diesem die globale OTF-Datei.

Zum Beispiel:
\begin{verbatim}
~/iopattern/rabbit> ./rabbit --anonptr -b /home/userxy/testtrace/iot.otf
\end{verbatim}

Als Ausgabe erhält man dann die folgenden 3 Dateien:
\begin{itemize}
\item Eine Datei namens \texttt{raw.dot}.
\item Eine Datei namens \texttt{cleaned.dot}. Im Vergleich zu \texttt{raw.dot}
        sind hier redundante Synchronisationsknoten entfernt.
\item Informationen, die nach \texttt{stdout} geschrieben werden.
\end{itemize}

\paragraph{Kurzanleitung von iopattern/rabbit}
\begin{verbatim}
 rabbit   -  analyze/classify I/O call within an Vampirtrace               
                                                                           
 Options:                                                                  
  -h,--help                 show this help message                         
  -m,--monitor <directory>  monitor only files in this directory           
                            can be given more than once                    
  -b,--barrier              sync I/O operations with blocking MPI events   
  --mask                    put mask over generated tokens before handing  
                            it over to the compression algorithm           
  --dumpall                 dump all generated tokens                      
  --anonptr                 use anonymous pointers where appropriate       
  --nocompress              don't compress graph during construction       
  --graphgen                dump dot file after each generation step       
  -g,--group <name>         sets the function group to be inspected,       
  -f,--func <name>          sets the function name to be inspected,        
                            option can be given more than once!            
  --plaindump               dumps the raw data for each call        
\end{verbatim}
%% Benutzung von iopattern2dios %%
\subsection{Benutzung von iopattern2dios} % erste Version fertig
\paragraph{Kurzanleitung}
\begin{verbatim}
Usage: python iopattern2dios.py [options]

Options:
  -h, --help                            show this help message and exit
  --encoding=ENCODING                   Encoding of the XML-file.
  -v, --verbose                         Turn verbosity on.
  -f INPUTDATA, --file=INPUTDATA        Path to the input-file.
  -o OUTFILE, --outfile=OUTFILE         Path to the output-file.

The default encoding is ISO-8859-1.

INPUTDATA must be a file, which contains the information, which is written
to stdout when running iopattern.

The format of OUTFILE is XML.
\end{verbatim}
\paragraph{Ausführliche Anleitung}~\\
Das Programm \emph{iopattern/rabbit} liefert drei verschiedene Ausgaben:
\begin{itemize}
\item Eine Datei namens \texttt{raw.dot}.
\item Eine Datei namens \texttt{cleaned.dot}. Im Vergleich zu \texttt{raw.dot}
	sind hier redundante Synchronisationsknoten entfernt.
\item Informationen, die nach \texttt{stdout} geschrieben werden.
\end{itemize}
Die beiden \texttt{*.dot}-Dateien werden von \emph{iopattern2dios} nicht 
benötigt, sondern nur das, was nach \texttt{stdout} geschrieben wird. 
Diese Ausgabe muß als Datei vorliegen, zum Beispiel durch Umleitung von 
\texttt{stdout} in eine Datei beim Ausführen von \emph{iopattern/rabbit}.

Zum Ausführen von \emph{iopattern2dios} übergibt man die Datei 
\texttt{iopattern2dios.py} dem Programm Python:
\begin{verbatim}
 python iopattern2dios.py
\end{verbatim}
Befindet man sich nicht in dem Verzeichnis, in dem die Datei 
\texttt{iopattern2dios.py} liegt, so muß man die Datei mit dem vollständigen
Pfad übergeben. 

Werden keine Optionen angegeben, so geht \emph{iopattern2dios} davon aus,
daß die Ausgabe des Programms \emph{iopattern/rabbit} in der Datei 
\texttt{input} gespeichert ist, die sich im aktuellen Verzeichnis 
befindet. Die Ausgabe von \emph{iopattern2dios} wird dann in der 
Datei \texttt{out.xml} im selben Verzeichnis mit dem Encoding ISO-8859-1 
gespeichert.

Einen Überblick über die möglichen Optionen von \emph{iopattern2dios} erhält
man, wenn man das Programm mit der Option \texttt{-h} bzw. \texttt{--help}
aufruft.
\begin{verbatim}
 python iopattern2dios.py --help
\end{verbatim}
Folgende Optionen sind möglich:\\~\\
\renewcommand{\arraystretch}{1.5}
\begin{longtable}{|l|l|p{0.4\textwidth}|}
\hline
\textbf{Langform} & \textbf{Kurzform} & \textbf{Erklärung}\\
\hline
\texttt{--help} & \texttt{-h} & Zeigt die in der Kurzanleitung dargestellte
				Übersicht über die möglichen Optionen. \\
\texttt{--encoding=<Kodierung>} &  & Gibt die Kodierung der XML-Datei, die
				die Ausgabe des Programms darstellt, an. 
				Ist keine Kodierung angegeben, so wird die 
				Kodierung ISO-8859-1 verwendet.\\
\texttt{--verbose} & \texttt{-v} & Bei Verwendung dieser Option wird 
				\emph{iopattern2dios} gesprächig und teilt mit,
				welche Optionen gewählt wurden, welche Datei 
				als Eingabe-Datei verwendet wird, in welche 
				Datei die Ausgabe geschrieben wird, welche 
				Kodierung benutzt wird und welche 
				Zeilen aus der Eingabe-Datei nicht verarbeitet
				wurden.\\
\texttt{--file=<Eingabe-Datei>} & \texttt{-f <Eingabe-Datei>} & Mit dieser 
				Option kann der Name der Eingabe-Datei von 
				\texttt{input} auf einen beliebigen Wert
				geändert werden. Liegt die Eingabe-Datei in 
				einem anderen als dem aktuellen Verzeichnis,
				so gibt man vor dem Dateinamen den 
				vollständigen Pfad an.\\
\texttt{--outfile=<Ausgabe-Datei>} & \texttt{-o <Ausgabe-Datei>} & Mit dieser 
				Option kann der Name der Ausgabe-Datei von 
				\texttt{out.xml} auf einen beliebigen Wert 
				geändert werden. Liegt die Ausgabe-Datei in 
				einem anderen als dem aktuellen Verzeichnis,
				so gibt man vor dem Dateinamen den 
				vollständigen Pfad an.\\
\hline
\end{longtable}
~\\
Die Ausgabe-Datei hat das richtige Format, um als Eingabe-Datei für das 
Programm \emph{DIOS} zu fungieren.
%%% Wichtige Hinweise %%%
\section{Wichtige Hinweise} % in erster Version fertig
\emph{iopattern2dios} ist in der Programmiersprache Python geschrieben. 
Die in der Datei \texttt{dios.dtd} definierten Tags und Attribute sind 
in \emph{iopattern2dios} implementiert. Es ist möglich und auch 
wahrscheinlich, dass die Ausgabe von \emph{iopattern/rabbit} 
Funktionen enthält, die 
nicht in \texttt{dios.dtd} definiert sind. In diesem Fall wird eine 
Warnung auf \texttt{stdout} geschrieben.
Ferner nimmt \emph{iopattern2dios} an, dass das Argument jeder Funktion in der 
Eingabe-Datei eine durch Komma getrennte Liste aus Schlüssel-Wert-Paaren ist, 
bei denen die einzelnen Schlüssel-Wert-Paare durch einen Doppelpunkt den 
Schlüssel vom Wert trennen.
Dies ist jedoch nicht immer der Fall. So kann die Ausgabe von 
\emph{iopattern/rabbit} zum Beispiel die Funktion 
\begin{verbatim}
  barrier_sync( 4)
\end{verbatim}
enthalten. Wie man sieht, ist hier das Argument kein Schlüssel-Wert-Paar.
Auch ist die Funktion \texttt{barrier\_sync} nicht in \texttt{dios.dtd}
definiert.
In diesem Fall verwandelt \emph{iopattern2dios} die Information wie folgt in
XML:
\begin{itemize}
\item Die Funktion \texttt{barrier\_sync} wird zu einem Tag. 
\item Die $4$ wird als Wert dem Attribut \verb="unknown"= zugeordnet.
\item[$\Rightarrow$] \verb+<barrier_sync unknown="4" />+
\end{itemize}
Des Weiteren werden zwei Warnungen nach \texttt{stdout} geschrieben:
\begin{verbatim}
Warning: 4 is not a key-value object. Can not find delimiter ':' in this string.
Warning: The function barrier_sync is not defined in dios.dtd!
\end{verbatim}
Durch die nach \texttt{stdout} geschriebenen Warnungen ist man immer informiert,
welche Informationen der Eingabe-Datei nicht in \texttt{dios.dtd} definiert 
sind und eventuell Probleme bereiten könnten. Warnungen werden übrigens auch
dann ausgegeben, wenn die Verbose-Option nicht benutzt wird.
\paragraph{Mapping}~\\
Dateien werden in einem bestimmten Modus geöffnet, z.\,B.
lesend oder schreibend. Üblicherweise wird dieser Modus durch einen 
Buchstaben (z.\,B. \emph{r} oder \emph{w}) angegeben. In der Ausgabe von
\emph{iopattern/rabbit} wird dieser Modus jedoch durch Zahlen kodiert, da dies 
\emph{VampierTrace} so macht. Diese Zahlen müssen wieder in Buchstaben 
zurückverwandelt werden. Dies ist in \emph{iopattern2dios} nach folgendem 
Schema implementiert:
\renewcommand{\arraystretch}{1.0}
\begin{center}
\begin{tabular}{ll}
1 & \emph{r}\\
2 & \emph{w}\\
4 & \emph{a}
\end{tabular}
\end{center}
Enthält die Eingabe-Datei an der entsprechenden Stelle eine andere Zahl als
die in der obigen Tabelle erwähnten, so wird eine Warnung nach \texttt{stdout}
geschrieben.
%%% Weitere in diesem Zusammenhang möglicherweise nützliche Programme %%%
\section{Weitere in diesem Zusammenhang möglicherweise nützliche Programme}
\begin{itemize}
\item Um sich die Struktur und den Inhalt einer XML-Datei gut anschauen zu 
	können, ist das Programm \emph{treeline} hilfreich.
\item Um sich den Inhalt der beiden \texttt{*.dot}-Dateien, die 
	\emph{iopattern/rabbit} als Ausgabe liefert, anschauen zu können, 
	ist das Programm \emph{Graphviz} hilfreich.
\end{itemize}
\end{document}

